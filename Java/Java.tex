\documentclass[11pt,twoside,a4paper]{report}

\usepackage{listings}
\usepackage{amssymb} %For Math symbols
\usepackage{textcomp}

\title{Java Notes}
\author{Amir Nourinia}

\begin{document}
\maketitle
\tableofcontents
\chapter{Introduction}
\section{Keywords}
The most important thing is that class and main keywords are both lowercase! :D

Moving on, all the keywords are lowercase. For a list you can search online. The wikipeida and w3schools or oracale have a list you can use as a reference.
\section{Difference and Similarties Between Java and C++}
First of all Java is completely OOP. Meaning even the main function gets wrapped inside a Class.
We mark the access modifer of the wrapper function for the main with public (lowercase). And also, we mark
the main function with:

\begin{lstlisting}
        static public void main(String[] args){
            // DoSomething
         }
        \end{lstlisting}

Note that we use barackets for Array declartaion, however, we use them after the type. In previous example the array declaration is done
with \verb|String[]|.

Java also uses semicolon \verb|;| to declare the end of an statement.

\section{Primative Data types}
\verb|char, byte, short, int, long, float, double|. These are the primative data types in Java. All of them also have a Class represetnation
which complements their functionality. For example: int\textrightarrow Int.

We can add suffixes to force the data type and be explicit about it. float - f, double - d, long - L.
For calculations requiring floating points it is better that we use Double which is much efficent in Java regardless of the fact that it has double precision meaning it occupies double the size
of a float. Also, as you know the double and float are not accurate representing the real numbers. For this reason there is a class \dots

Also, the primitive data types are passed by value. Everything else in Java, is passed by reference.

To convert String to int, long etc., we use the Class type of data and .parseType(String arg) method.
\section{String}
Notice the capital S. The \verb|String| type is a class in Java. And strings are immutable in Java, meaning each time we modify one we are actually creating a new one in memory.
This makes String manipulation inefficient in Java. For this reason we use String Buffers.

\section{Expression vs Statement.}
Expression is part of the code that will result in a value.
Statement is a command line ending with \verb|;|. This is same as the C++.

\section{Methods}
Just Like C++ we can not declare or define another method inside another method. In java however, we can use a method before we define or declare it. But, it has to be defined some where in the code.

In method overloading, unline C++, in Java chaging the return type does not change the method signature. For this reason, the only way to change method signature is using the input types and numbers.

\section{Control flow}
switch(switchValue) in swich in the cases we use a break; to make sure that omly one case is gets executed.
\begin{lstlisting}
            switch(switchValue){
                case 1:
                 //DoSomething
                 break;
                case 2:
                // case2
                break;

                case 3:case 4:case 5:
                //multiple cases
                break;

                default:
                // default case
                break; /* This break is not necessary, 
                        it is the last one. */
            }
            \end{lstlisting}

So use break; to make sure the code works properly. Otherwise it can continue to other cases after the intended case.

\section{Print out and read user input}
Note the that the \verb|new| here returns an address to the memory. Unlike C++, we don't explicitly assign this value to a pointer type.
However, the variable scanner, will be a refrence (implicit pointer) type and will be pointing to a part of heap memory.
\begin{lstlisting}
            // print to stdout.
            System.out.printl(String arg);
            // allows getting input from user.
            Scanner scanner = new Scanner(System.in);
            // To read a line
            String name = scanner.nextLine();

            /* To read an int, after this call we 
            always call nextLine() to handle 
            entered breakline character, which 
            we enter as the user. */
            int age = scanner.nextInt();
            scanner.nextLine();
            

            /* Check if the next input in buffer is a 
            valid Int without discarding it. */
            scanner.hasNextInt();
            //when done with scanner\dots
            scanner.close();
            \end{lstlisting}

\chapter{OOP}
State in an object is stored in fields. Fields are the counter parts of memeber variables in C++.
A class variable in OOP, however, is any variable declared with the static modifier of which a single copy exists, regardless of how many instances of the class exist.

As a convention, the first letter in java classes should be a uppercase letter to make it easy to identify the Class type from their respective objects.

Java Classes get saved in a .java format text file. Each class gets saved in a package folder which, we can see that package name in the first line of the class. It starts with the keyword package.

The public access modifier for class gives access to others to the class. Meaning, the other classes or developers can access the class based on this modifier. 
The public means unrestricted access to the class.
The access modifier in Java are as follows:
\begin{itemize}
    \item public - unrestricted access
    \item private - no other class can access the class
    \item protected - allows classes in this package to access your class
\end{itemize}

We can also omit the access modifier if we want.

The Fields also get access modifier just like C++. The same access modfier keywords as class.
The default access modifier for fields if ommited is private. If an access modifier for fields is more restrictive than its encapsulating class,
the field's access modifier gets precedence.
\begin{lstlisting}

    public class Car {
        // Accessable only by the class
        private String model;
        private String color;

        // Accessable by anyone!
        public String owner;

        // Set owner, accessable by anyone.
        public void setOwner(String owner){
            this.owner = owner;
        }

    }

\end{lstlisting}



Java adds some basic functionality to each Class we create. An object of any class in Java is child of Object class which implments these functionalities.

Also, it is important that in Java declaring of a variable should be its initalization. RAII (Resource allocation is initalization).
This means, if we write the code: Car porsche; This would create null referene. We have to assign an address to it explicitly. So, we fix the code to:  Car porsche = new Car();
this assigns the reference the address in the heap. Actually, the compiler will not let us to compile a code with uninitialized refrence.

\end{document}