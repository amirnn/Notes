\documentclass[11pt,twoside,a4paper]{report}
\usepackage{listings}
\usepackage{hyperref}
\hypersetup{
    colorlinks=true,
    linkcolor=blue,
    filecolor=magenta,      
    urlcolor=cyan,
}
\author{Amir Nourinia}

\title{C++ Notes}

\begin{document}
\maketitle
\tableofcontents
\chapter{Primitive Types}

\section{string}
C++ introduces its own class for C++. It can be included using \#include <string>.
Note that string starts with small letter.
"a" is type of char* which is a pointer to character type also known as c style string.
however 'a' is char type which is a character. This is important and also different from Java.


\section{functions}
In functions note that the default value for the arguments should be defined in functione declaration. This is specially important in classes when you can easily make the mistake of defining default values in fucntion definition as well.
There can be as many declarations for function but there should be only one definition (implmentation).

Best practice for returning a value by reference. Taken from \href{https://stackoverflow.com/questions/4986673/c11-rvalues-and-move-semantics-confusion-return-statement}{Here}.
i.e. just as you would in C++03. tmp is implicitly treated as an rvalue in the return statement. It will either be returned via return-value-optimization (no copy, no move), or if the compiler decides it can not perform RVO, then it will use vector's move constructor to do the return. Only if RVO is not performed, and if the returned type did not have a move constructor would the copy constructor be used for the return.
\begin{lstlisting}


std::vector<int> return_vector(void)
{
    std::vector<int> tmp {1,2,3,4,5};
    return tmp;
}

std::vector<int> rval_ref = return_vector();

\end{lstlisting}

\chapter{Classes}
\section{Initialization}
Initialization of class memeber variables is important since, if we don't initialize them they are left with a random meaning less number. For this reason either give the values a initial (default) value in the declaration in class body
or initialize them inside a Member initializer list which is the place where non-default initialization of these objects can be specified.
\begin{lstlisting}
    Class_Name(): memeber{val},..{}
\end{lstlisting}
\section{Constructors}
Taken from: \href{https://en.cppreference.com/w/cpp/language/constructor}{Here} \\
Constructors have no names and cannot be called directly. They are invoked when initialization takes place, and they are selected according to 
the rules of initialization. The constructors without explicit specifier are converting constructors. The constructors with a constexpr specifier 
make their type a LiteralType. Constructors that may be called without any argument are default constructors. Constructors that take another object of the same type as the argument are copy constructors and move constructors. 

Note that we can also call the constructors of the base classes in dervied classes's constructors.

\begin{lstlisting}
// Note that I am calling the base constructor
derived(args...) : base(args...), value1{arg}, 
value2{arg}
{
    \\doSomething
}
\end{lstlisting}

For default contructor it does not matter if we call the it or not since, C++ will understand this by itself.
String myString; is equal to String myString();

\subsection{Defualt Constructor}
If no user-declared constructors of any kind are provided for a class type (struct, class, or union), the compiler will always declare a default constructor as an inline public member of its class.
If some user-declared constructors are present, the user may still force the automatic generation of a default constructor by the compiler that would be implicitly-declared otherwise with the keyword default. (since C++11)
The implicitly-declared (or defaulted on its first declaration) default constructor has an exception specification as described in dynamic exception specification (until C++17)exception specification (since C++17)
See \href{https://en.cppreference.com/w/cpp/language/default_constructor}{Here.}
This means if we declare our own custome constructor of anykind the defualt constructor will be omitted by the compiler. But again, we can force its generation by the compiler or, we can declare it ourselves.


\subsection{Converting constructor}
Taken from: \href{https://en.cppreference.com/w/cpp/language/converting_constructor}{Here} \\
A constructor that is not declared with the specifier explicit and which can be called with a single parameter \(until C++11\) is called a converting constructor.
Unlike explicit constructors, which are only considered during direct initialization \(which includes explicit conversions such as static_cast\), converting constructors are also considered during copy initialization, as part of user-defined conversion sequence. 

\section{Destructors}
\subsection{Virtual Destructor}
Deleting a derived class object using a pointer to a base class that has a non-virtual destructor results in undefined behavior. To correct this situation, the base class should be defined with a virtual destructor.
For example, following program results in undefined behavior. \href{https://www.geeksforgeeks.org/virtual-destructor/}{Here}

Also, note that destructor can be pure virtual function. \href{https://en.cppreference.com/w/cpp/language/abstract_class}{Read Here for more explianation.}

Read \href{https://en.cppreference.com/w/cpp/language/virtual}{Here} about the virtual destructors.
Even though destructors are not inherited, if a base class declares its destructor virtual, the derived destructor always overrides it. This makes it possible to delete dynamically allocated objects of polymorphic type through pointers to base.
Moreover, if a class is polymorphic (declares or inherits at least one virtual function), and its destructor is not virtual, deleting it is undefined behavior regardless of whether there are resources that would be leaked if the derived destructor is not invoked.
A useful guideline is that the destructor of any base class must be public and virtual or protected and non-virtual.

\section{Inheritance}
A short intro to inhertance with different access specifiers (public, protected, and private) is \href{https://www.tutorialspoint.com/cplusplus/cpp_inheritance.htm}{here}

In class inhertance have in mind that the private memebers of the base class are only availabe by that class. No other dervied classes can access them directly. However, they can access the memebr variables using calls (set, get).
\section{Abstract Classes}
Abstract classes are used to represent general concepts (for example, Shape, Animal), which can be used as base classes for concrete classes (for example, Circle, Dog).
No objects of an abstract class can be created (except for base subobjects of a class derived from it) and no non-static data members of an abstract class can be declared.
Abstract types cannot be used as parameter types, as function return types, or as the type of an explicit conversion (note this is checked at the point of definition and function call, since at the point of function declaration parameter and return type may be incomplete)
Pointers and references to an abstract class can be declared. \href{https://en.cppreference.com/w/cpp/language/abstract_class}{Read more here.}

Note that, we can provide a defintion for a pure virtual function outside the class body. We can not however do this inside the class becasue we can use =0 and function definition at the same time.

Also, pay attention to the example provided by cppreference page. \href{https://en.cppreference.com/w/cpp/language/abstract_class}{Here}
As you can see there we can make a turn a virtual function in a base concrete class into a pure one and create another abstract interface class.

When inheriting from abstract classes we can call the constructor that we provided for the abstract class in the derived class's constructor. 
The Abstract classes do not have a default constor (? Or if they do it is not callable, and hence, no object of abstract type can be generated.); So, if we have not declare any different constructors for the abstract class we do not need to call anything when defining the constructor for the derived classes.
\subsection{virtual keyword.}
Read this page defenitly. \href{https://en.cppreference.com/w/cpp/language/virtual}{Here}
Take a close look at: Then this function in the class Derived is also virtual (whether or not the keyword virtual is used in its declaration) and overrides Base::vf (whether or not the word override is used in its declaration).
Base::vf does not need to be accessible or visible to be overridden. (Base::vf can be declared private, or Base can be inherited using private inheritance. Any members with the same name in a base class of Derived which inherits Base do not matter for override determination, even if they would hide Base::vf during name lookup.) 

\subsection{Virtual Functions}
A virtual function is a member function which is declared within a base class and is re-defined(Overriden)
by a derived class. When you refer to a derived class object using a pointer or a reference to the base class,
you can call a virtual function for that object and execute the derived class’s version of the function.
See \href{https://www.geeksforgeeks.org/virtual-function-cpp/}{Here}

\subsection{Pure Virtual Functions}
Sometimes implementation of all function cannot be provided in a base class because we don’t know the implementation. Such a class is called abstract class. For example, let Shape be a base class. We cannot provide implementation of function draw() in Shape, but we know every derived class must have implementation of draw(). Similarly an Animal class doesn’t have implementation of move() (assuming that all animals move), but all animals must know how to move. We cannot create objects of abstract classes.

A pure virtual function (or abstract function) in C++ is a virtual function for which we don’t have implementation, we only declare it. A pure virtual function is declared by assigning 0 in declaration. See the following example.
See \href{https://www.geeksforgeeks.org/pure-virtual-functions-and-abstract-classes/}{Here}

\subsection{Can virtual functions be private in C++?}
In C++, virtual functions can be private and can be overridden by the derived class. See \href{https://www.geeksforgeeks.org/can-virtual-functions-be-private-in-c/}{this}
Note that this behavior is totally different in Java. In Java, private methods are final by default and cannot be overridden (See \href{https://www.geeksforgeeks.org/can-override-private-methods-java/}{this})

\section{const keyword on member functions (methods) and constructor functions (methods)}
A constructor can not be marked as const using const type qualifier. Which makes sense since we are going to modeify or initilize the memory where the class is being held in.
Also, after using a const type qualifier on a memeber function (which is by nature not constructors) we have to use the same const in their definition since the const becomes part of the
identifier.

\begin{lstlisting}
    public:
     void print() const;

     //note the default value in function declaration.
     void split(std::string delmiter = " ")

     //later in definition
     void String::print() const {
         \\do something
         }
\end{lstlisting}

\section{Class Scope Identifier in Function Identifier}
On the last code example, notice the function signature "void String::print() const".\\
And note that the return type comes before the scope indentifier "String::" This makes sense since we are looking for the symbol or the name, print, in the String Class scope,
so, the print keyword exists only in String::print form.

\section{static keyword}
There are two usages of static Memory and for Memeber Variables which also in reference to the memory duration of the variable.
\subsection{Memory Duration}
See \href{https://en.cppreference.com/w/cpp/language/storage_duration}{Here} for memory explianations.
The static specifier is only allowed in the declarations of objects (except in function parameter lists), declarations of functions (except at block scope), 
and declarations of anonymous unions. When used in a declaration of a class member, it declares a static member.
When used in a declaration of an object, it specifies static storage duration (except if accompanied by thread\_local).
When used in a declaration at namespace scope, it specifies internal linkage.

Read about the linkage it can come in handy.
\subsection{Memeber Variable}
See \href{https://en.cppreference.com/w/cpp/language/static}{here.}
Inside a class definition, the keyword static declares members that are not bound to class instances.
Static members of a class are not associated with the objects of the class: they are independent variables
with static or thread (since C++11) storage duration or regular functions. The static keyword is only used 
with the declaration of a static member, inside the class definition, but not with the \textbf{definition} of that static member: 

So, we do not use static in the definition of that memeber (variable or function).

\begin{lstlisting}
// declaration (uses 'static')
class X { static int n; };
// definition (does not use 'static')
int X::n = 1;              
\end{lstlisting}

The declaration inside the class body is not a definition and may declare the member to be of incomplete type (other than void), including the type in which the member is declared: 
\begin{lstlisting}
struct Foo;
struct S
{
   static int a[]; // declaration, incomplete type
   static Foo x;   // declaration, incomplete type
   static S s;     // declaration, incomplete type
                   // (inside its own definition)
};
 
int S::a[10]; // definition, complete type
struct Foo {};
Foo S::x;     // definition, complete type
S S::s;       // definition, complete type
\end{lstlisting}


\chapter{Streams}
std::cin and std::cout are objects of type std::istream which can be passed around. However, the istream move and copy constructor are deleted so we can only send them by reference.

\begin{lstlisting}
    void readFromStream(std::istream&);
    // later in usage:
    readFromStream(std::cin);
\end{lstlisting}


\end{document}